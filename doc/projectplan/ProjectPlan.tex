% !TeX program = xelatex
% !TeX encoding = utf8
% !TeX root = ProjectPlan.tex
% vim: set ts=2 sw=2 noet linebreak spell:

\documentclass[a4paper, twosided, 11pt]{scrartcl}

%% Language configuration
\usepackage{polyglossia}
\setdefaultlanguage{english}

%$ Specify where floatings go and make nicer looking tables
\usepackage{float}
\usepackage{array}
\usepackage{booktabs}
\usepackage{tabularx}

%% graphics
\usepackage{graphicx}

%% Set up font
\usepackage[T1]{fontenc}
\usepackage[usefilenames, DefaultFeatures={Ligatures=Common}]{plex-otf}
\renewcommand*\familydefault{\sfdefault}

%% Header and Footers
\usepackage[automark]{scrlayer-scrpage}
\ihead{Project Plan}
\chead{}
\ohead{\headmark}

\ifoot{}
\cfoot{}
\ofoot{\pagemark}

%% Access metadata
\usepackage{titling}

%% Pretty pictures
\usepackage{tikz}
\usetikzlibrary{patterns}

%% gantt configuration
\usepackage{pgfgantt}
\usepackage[]{geometry}

%% Landscape pages
\usepackage{pdflscape}

% Metadata
\title{Project Plan}
\author{Naoki Pross, Sara Cinzia Halter}


% Document
\begin{document}
\begin{titlepage}
	\begin{flushright}
		\vspace{5cm}
		{\Huge \bfseries \thetitle} \\
		\vspace{5mm}
		{\LARGE Project: \textit{Fading illustration with SDR}} \\
		\vspace{5mm}
		{\LARGE \bfseries Version 1.0}
	\end{flushright}
\end{titlepage}

\clearpage
\clearpage
\tableofcontents

\vfill
{
	\renewcommand{\arraystretch}{2}
	\begin{tabularx}{\textwidth}{lp{.2\textwidth}X}
		                   & \bfseries Date & \bfseries Signature\\
		Halter Sara Cinzia & \hrulefill & \hrulefill \\
		Pross Naoki Sean   & \hrulefill & \hrulefill \\
	\end{tabularx}
}
\clearpage

\section{Introduction}

For the semester thesis at the Eastern Switzerland University of Applied Sciences (OST) it has been requested to create a demonstrative setup to show the fading effect, which is present in real world wireless communication systems.
The device is intended to be used for pedagogical purposes such as to show the effect at the Open Days or for demonstrations during future lectures on fading channels.

\section{Task Description}

\begin{figure}[h]
	\centering
	\begin{tikzpicture}[
			cpoint/.style = {
				inner sep = 0,
				outer sep = 0,
			},
			computer/.pic = {
				% display
				\node[
					draw, thick,
					rounded corners = 2pt,
					minimum height = 10mm,
					minimum width = 15mm,
				] (-screen) {};
				% stand
				\draw[thick] (-screen.south) ++(-2mm,0) 
					to[out = -90, in = 60] ++(-1mm, -3mm) 
					to ++(6mm,0)
					to[out = 120, in = -90] ++(-1mm, 3mm);
			},
			antenna/.pic = {
				\draw[very thick] (0,0) -- ++(2mm, 3mm) -- ++(-4mm,0) -- cycle;
				\draw[very thick] (0,0) -- ++(0,-5mm) node[cpoint] (-mast) {};
				\draw[thick] (0,0) -- ++(0,3mm);
				\node[outer sep = 2mm] (-center) at (0,-1mm) {};
			},
		]

		\begin{scope}[shift = {(-10, 2)}]
			\pic (txPC) {computer};
			\node at (txPC-screen) {\textbf{TX}};
			\node[draw, thick, right = 5mm of txPC-screen, yshift = -6mm] (txSDR) {SDR};
			\pic[above = of txSDR.north east, xshift = -3mm] (txA) {antenna};
			\draw[very thick]
				(txPC-screen.east) ++(0,-.2) to[out = 0, in = 180] (txSDR.west)
				(txA-mast) -- (txA-mast |- txSDR.north);

			\node[below = 5mm of txPC-screen.south] {Stationary sender};
		\end{scope}

		\begin{scope}
			\pic (rxPC) {computer};
			\node at (rxPC-screen) {\textbf{RX}};
			\node[draw, thick, left = 5mm of rxPC-screen, yshift = -6mm] (rxSDR) {SDR};
			\pic[above = of rxSDR.west, xshift = -15mm] (rxA) {antenna};
			\draw[very thick]
				(rxPC-screen.west) ++(0,-.2) to[out = 180, in = 0] (rxSDR.east)
				(rxA-mast) to[out = -90, in = 0] ++(0,-1)
					to[out =  180, in = -120] ++( -4mm,  2mm)
					to[out =   50, in =   60] ++(  8mm, -1mm)
					to[out = -120, in = -120] ++( -8mm,  0mm)
					to[out =   30, in =  150] ++(  7mm,  1mm)
					to[out =  -30, in =  180] (rxSDR.west)
				;
			\node[below = 5mm of rxPC-screen.south] {Moving Receiver};
		\end{scope}

		\draw[line width = 1mm, draw = red!50!white, -latex] 
			(txA-center) -- node[midway, above, sloped] {LOS path} (rxA-center);

		\begin{scope}[
			ultra thick, draw = blue!50!white, -latex
		]
			\draw (txA-center) -- ++(5,0) node[above] {Reflected paths} -- (rxA-center);
			\draw (txA-center) -- ++(3,1) -- (rxA-center);
			\draw (txA-center) -- ++(1,-2) -- (rxA-center);
		\end{scope}
	\end{tikzpicture}
	\caption[
		Sketch of the setup that will be modelled and implemented.
	]{ Sketch of the setup that will be modelled and implemented.
		The model will need to be adjusted depending on whether there is a line of sight (LOS) between the sender and receiver.
	}
\end{figure}

The scope of the project is to realize a demonstration of a fading channel using a software defined radio (SDR).
Out of the many types of fading effects that exist only small scale fading effects ought to be shown in the demonstration, specifically multi path propagation fading is of interest.
The project requirements that must be fulfilled are thus:
\begin{itemize}
	\item Understand of one or more mathematical models of the fading effect.
	\item Evaluate a suitable development environment for the SDR.
	\item Develop of a signal processing chain for the SDR transmitter and receiver.
	\item Develop of an interface to vary the parameter of the transmission, such as the modulation scheme.
	\item The demonstration should work with both stationary and moving receivers.
\end{itemize}

\section{Development plan}

The development of the project will be carried out roughly in three phases, of which the first two will start in parallel.
\begin{enumerate}
	\item Develop an understanding of fading and how to work with SDR devices.
	\item Create a basic TX -- RX line \emph{without} a fading channel model.
	\item Integrate the fading channel model into the prototype.
\end{enumerate}

\subsection{SDR Device}
Finding the right software tool, like GNURadio or Matlab, for the Project and learn how the program works. The same with the Hardware, is the \textit{USRP B210} the best option or is there an other SDR, which fits better too our requirements. 

\subsection{Prototype}
Create the first Prototype line TX-RX without any effects. Including some simulation on that setup first and then tests and measurements on the Hardware. 
After a successful set up, some possible variable parameters will be included. 

\subsection{Theory of fading channels}
When the Prototype works the integration of the fading channel models starts, including simulations and tests on the Hardware. 
At the end comes the same with a movable Reviser. Walking will be used to demonstrate that on a scenario. If there will be some time left also with the help of a car.

\newpage
\section{Milestones}

\begin{table}[h]
	\centering
	\caption{Milestones of the project}
	\renewcommand\arraystretch{1.2}
	\begin{tabularx}{\linewidth}{l c X}
		\toprule
		\bfseries Name & \bfseries Due date & \bfseries Description \\
		\midrule
		Project plan & Week 40 & 
			Finalization of this document. \\

		Working SDR TX -- RX  & Week 44 &
			Completion of an RX -- TX line on SDR with variable parameters for configuration. \\

		Working fading TX -- RX & Week 49 &
			Both the simulated and the physical transmission lines work and it is possible to observe the consequences of fading.\\
			
		Abstract & 17 Dec. 2021 &
			The abstract is handed in. \\

		Documentation & Week 50 &
			The documentation is complete both on the theory and practical sides. \\

		Presentation & 23 Dec. 2021 &
			Presentation of the project on Campus. \\
		Submission & 24 Dec. 2021 & --- \\
		\bottomrule
	\end{tabularx}
\end{table}

\appendix

\clearpage
\newgeometry{vmargin = 2.5cm, hmargin = 1.5cm} 
\begin{landscape}
	\begin{figure}[h]
		\caption{Project schedule (Gantt diagram)}
		\begin{ganttchart}[
			expand chart = \linewidth,
			% use dates
			time slot format = isodate,
			% set up canvas
			canvas/.append style = {thick},
			title/.style = {draw, thick},
			% set grid
			vgrid = {*6{dotted}, *1{blue!50!black, thick, dotted}},
			% set height and width
			x unit = 2.5mm,
			y unit chart = 12mm,
			% put everything in the chart 
			inline,
			% modify positioning of elements
			group top shift = .6,
			bar top shift = .1,
			bar height = .4,
			milestone top shift = .2,
			milestone height = .3,
			% colored elements
			bar/.append style={
				fill = teal!30!white,
			},
			bar incomplete/.append style={
				fill = white
			},
			progress label text=\relax,
			% modify stile of elements
			bar/.append style = {
				thick,
			},
			bar inline label node/.append style = {
				font = \footnotesize,
			},
			milestone/.append style = {
				fill = red!50!white, thick,
			},
			milestone inline label node/.append style = {
				font = \footnotesize\itshape,
				below = 3mm, xshift = 2mm,
				anchor = north east,
				fill = white,
			},
		]{2021-09-20}{2021-12-26}

			\gantttitlecalendar{month = name, week = 38} \\

			\ganttgroup{Research}{2021-09-21}{2021-11-28}
			\ganttnewline
			
			\ganttbar[progress=20]{Theory}{2021-09-22}{2021-11-28}
			\ganttnewline

			\ganttbar [progress=25]{Software Toolboxes}{2021-09-27}{2021-10-31}
			\ganttnewline
						
			\ganttbar[progress=6]{Hardware}{2021-10-04}{2021-10-31}
			\ganttnewline			

			\ganttgroup{Prototype}{2021-10-04}{2021-10-31}
			\ganttgroup{Demonstration}{2021-11-03}{2021-12-05}
			\ganttnewline

			\ganttbar[progress=0]{Simulation}{2021-10-04}{2021-10-18}
			\ganttbar[progress=0]{Hardware}{2021-10-18}{2021-10-31}
			\ganttmilestone{Working SDR TX -- RX}{2021-11-01}
			\ganttbar[progress=0]{Fading model}{2021-11-03}{2021-11-19}
			\ganttbar[progress=0]{Measurements}{2021-11-20}{2021-12-05}
			\ganttmilestone{Working fading TX -- RX}{2021-12-06}
			\ganttnewline

			\ganttgroup{Documentation}{2021-09-21}{2021-12-20}
			\ganttnewline
			
			\ganttbar{Project Plan}{2021-09-21}{2021-10-04}
			\ganttmilestone{Submission}{2021-10-05}
			\ganttbar[progress=0]{Elevator Pitch Video}{2021-11-08}{2021-11-28}
			\ganttbar[progress=0]{Abstract}{2021-12-02}{2021-12-09}
			\ganttmilestone{Abstract Draft}{2021-12-10}
			\ganttnewline

			\ganttbar[progress=0]{Introduction and Task Desc.}{2021-09-27}{2021-10-24}
			\ganttbar[progress=0]{Methods (SW and HW)}{2021-10-27}{2021-11-16}
			\ganttbar[progress=0]{Results}{2021-11-20}{2021-12-11}
			\ganttmilestone{Ready}{2021-12-12}
			\ganttbar[progress=0]{Review}{2021-12-14}{2021-12-22}
			\ganttmilestone{Submission}{2021-12-23}
			\ganttnewline

			\ganttbar[progress=0]{Theory}{2021-10-07}{2021-12-04}
			\ganttbar[progress=0]{Slides}{2021-12-06}{2021-12-22}
			\ganttmilestone{Presentation}{2021-12-23}
			\ganttnewline
		\end{ganttchart}
	\end{figure}
\end{landscape}
\restoregeometry
\end{document}
