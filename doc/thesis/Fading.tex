% !TeX program = xelatex
% !TeX encoding = utf8
% !TeX root = Fading.tex
% vim: set ts=2 sw=2 noet:

\documentclass[
	% show when a line is too long (drafting)
	overfullrule,
	% page size and margins
	paper = a4, twoside, openright, BCOR = 5mm,
	headinclude, footexclude,
	% font size
	fontsize = 11pt,
	% style of empty pages (after chapters)
	cleardoublepage = empty,
	% extra sections
	titlepage, abstract = on,
	% header and footers
	automark,
]{scrreprt}

%% Language configuration
\usepackage{polyglossia}
\setmainlanguage{english}

% Use a different font for german
\usepackage{xcolor}
\setotherlanguage{german}
\newfontfamily{\germanfont}{Roboto Condensed Light}

%% Custom packages
\usepackage{tex/docmacros}
\usepackage[draft]{tex/docstyle}

%% Pretty figures
\usepackage{circuitikz}  % Electric diagrams
\usepackage{pgfplots}    % Pretty plots
\usepackage{tikz}        % Pretty drawings
\usepackage{tikz-3dplot} % More dimensions!

\pgfplotsset{compat=newest}
\usetikzlibrary{
	external,
	calc,
	positioning,
	backgrounds,
	decorations.pathreplacing,
	calligraphy,
	decorations.markings,
	matrix,
	arrows,
	patterns,
}
% \tikzexternalize[
% mode=graphics if exists,
% figure list=true]
\pgfdeclarelayer{background}
\pgfdeclarelayer{foreground}
\pgfsetlayers{background,main,foreground}

\usepackage{graphicx}   % Include pictures
\usepackage{subcaption} % Subfigures

%% Placeholders
\usepackage{skeldoc}

%% Nicer tables
\usepackage{tabularx}
\usepackage{booktabs}

%% Writing units
\usepackage{siunitx}
% this one is for siunitx v3, debian and older systems
% still have only siunitx v2 installed
% \sisetup{uncertainty-mode = separate}
\sisetup{separate-uncertainty}

%% List of acronyms
% \usepackage{acronym}

%% Load bibliography 
\addbibresource{Fading.bib}

%% Recompute page margins
\KOMAoptions{DIV=default}

%% Metadata
\title{Multipath fading demonstration using software defined radio}
\author{Naoki Sean Pross \and Sara Cinzia Halter}
\date{Fall semester 2021}

\begin{document}

	\hypersetup{pageanchor = false}

	%% TODO: create a proper titlepage
	\maketitle
	% % vim: set ts=2 sw=2 noet:

\begin{titlepage}
	
%	{\large Multipath Fading Demonstration Platform using Software Defined Radio\par}
%	\vspace{1cm}
%	{\scshape Naoki Sean Pross \and Sara Cinzia Halter\par}
%	\vspace{1cm}
% 	{\scshape Semester Thesis\par}
%	{\scshape December 23, 2021\par}

%	TODO. Picture
%	\vfill
%	Field of Study \par
%	Wireless Communications
%	Advisor, Examiner \par
%	Prof. Dr. Heinz Mathis
%	OST Eastern Switzerland University of Applied Sciences, Rapperswil

	
\end{titlepage}


	\cleardoublepage
	\pagenumbering{roman}
	\setcounter{page}{1}

	\begin{abstract}
		%% TODO: write abstract
		\skelpar
	\end{abstract}

	\tableofcontents
	\cleardoublepage

	%% TODO: move in a separate file
	\chapter*{List of symbols}
	\noindent %
	\begin{tabularx}{\linewidth}{>{\(}c<{\)} X}
		\toprule
		% \textbf{Notation} & \bfseries Description \\
		% \midrule
		\multicolumn{2}{l}{\itshape Mathematics} \\
		\vec{v},\, \vec{m} & Vector quantity \\
		\phi^*             & Complex conjugate of \(\phi\) \\
		\midrule
		\multicolumn{2}{l}{\itshape Physical quantities} \\
		t      & Time \\
		T      & Period \\
		\tau   & Convolution time \\
		f      & Frequency in Hz \\
		\omega & Frequency in radians per second \\
		\Omega & Digital frequency in radians per sample \\
		\midrule
		\multicolumn{2}{l}{\itshape Modulation schemes} \\
		n(t)    & Additive white Gaussian noise (AWGN) \\
		f(t)    & Multiplicative fading noise \\
		p(t)    & Pulse function \\
		m(n)    & Message samples \\
		m(t)    & Message waveform \\
		\phi(t) & Carrier signal \\
		s(t)    & Signal waveform \\
		s(n)    & Signal samples \\
		r(t)    & Received signal \\
		r(m)    & Received samples \\
		\midrule
		\multicolumn{2}{l}{\itshape Channel modelling} \\
		h(t)       & LTI impluse response \\
		h(\tau, t) & LTV impluse response \\
		h(m)       & LTI discrete time impulse response \\
		h_l(m)     & LTV discrete time impulse response \\
		H(f)       & LTI Frequency response \\
		H(f, t)    & LTV Frequency response \\
		c_k        & Tap attenuation \\
		\tau_k     & Tap delay \\
		T_d        & Delay spread \\
		B_c        & Coherence bandwidth \\
		R_{xy}     & Correlation between \(x\) and \(y\) \\
		\bottomrule
	\end{tabularx}
	\cleardoublepage
	%TODO: Alphabetisch ortnen
	\chapter*{List of Acronyms}
	\noindent %
	\begin{tabularx}{\linewidth}{>{\(}c<{\)} X}
		\toprule
		% \textbf{Notation} & \bfseries Description \\
		% \midrule
		IoT & internet of things\\
		QAM & quadrature amplitude\\
		QPSK & quadrature phase shift keying\\
		PSK & phase shift keying\\
		SDR & software
defined radio\\
		BER & bit error rate\\
		AWGN & additive white Gaussian noise\\
		LTV & linear time-varying
system\\
		LTI & inear time invariant\\
		FIR & finite impulse
response (filter)\\
		CIR & channel impulse response\\
		WSSUS & Wide Sense Stationary Uncorrelated Scaterring\\
		WSS & Wide Sense Stationary\\
		NLOS & no line of sight\\
		LOS & line of sight\\
		GR & GNU Radio\\
		GRC & GNU Radio Companion\\
		DPG & Dear PyGUI\\
		IMGUI & immediate graphical user interface\\
		IP & Internet Protocol\\
		UDP & User Datagram Protocol\\
		API & Application Programming Interface\\
		PDP & Power delay profile \\
		MSB & most significant bit\\
		CAZAC& constant amplitude zero autocorrelation waveform\\

		\bottomrule
	\end{tabularx}
	\cleardoublepage
	

	\hypersetup{pageanchor = true}
	\pagenumbering{arabic}
	\setcounter{page}{1}
	\pagestyle{scrheadings}

	% vim: set ts=2 sw=2 noet:

\chapter{Introduction}

\section{Background}

Since decades now modern wireless devices have become so ubiquitous and are no longer employed under carefully chosen conditions. Cellphones and IoT devices are carried around by users and thus have to work in environments where reflexions are omnipresent. In order to efficiently develop such devices we need for mathematical models to simulate such environments.

	% vim: set ts=2 sw=2 noet spell:

\chapter{Theory}

\begin{figure}
	\centering
	\documentclass[tikz]{standalone}

\usepackage{roboto}
\usepackage{roboto-mono}
\usepackage{tikz}        % Pretty drawings
\usepackage{tikz-3dplot} % More dimensions!
\usetikzlibrary{
	external,
	calc,
	positioning,
	backgrounds,
	decorations.pathreplacing,
	calligraphy,
	decorations.markings,
	matrix,
	arrows,
	patterns,
}
\pgfdeclarelayer{background}
\pgfdeclarelayer{foreground}
\pgfsetlayers{background,main,foreground}

\begin{document}
\documentclass[tikz]{standalone}

\usepackage{roboto}
\usepackage{roboto-mono}
\usepackage{tikz}        % Pretty drawings
\usepackage{tikz-3dplot} % More dimensions!
\usetikzlibrary{
	external,
	calc,
	positioning,
	backgrounds,
	decorations.pathreplacing,
	calligraphy,
	decorations.markings,
	matrix,
	arrows,
	patterns,
}
\pgfdeclarelayer{background}
\pgfdeclarelayer{foreground}
\pgfsetlayers{background,main,foreground}

\begin{document}
\include{tikz/overview.tex}
\end{document}

\end{document}

	\caption{
		Block diagram for a general wireless communication system with annotated signal names. Frequency domain representations of signals use the uppercase symbol of their respective time domain name.
		\label{fig:notation}
	}
\end{figure}

In this section we will briefly give the mathematical background required by the modulation schemes used in the project. The notation used is summarised in figure \ref{fig:notation}. For conciseness encoding schemes and (digital) signal processing calculations are left out and discussed later. Thus for this section \(m_e = m\).

%% TODO: Par on notation m(n) = m(nT) = discrete time

\section{Quadrature amplitude modulation (\(M\)-ary QAM)}

\begin{figure}
	\centering
	% vim: set ts=2 sw=2 noet:

\begin{circuitikz}[
	]
	\matrix [
		row sep = 5mm, column sep = 7mm,
		nodes = {
			align = center,
			fill = white,
		},
	] {
		& \coordinate (vmi);
			& \node[twoportshape] (B2Li) {};
			&
			& \coordinate (mi);
			&
			& \node[mixer] (Mi) {};
			& \coordinate (si);
			\\
		\node[] (M) {\(m(n)\)};
			& \node[twoportshape] (BSp) {};
			&
			&
			& \node[twoportshape] (H) {};
			& \node[oscillator] (OSC) {};
			& \coordinate (phii);
			& \node[adder] (SUM) {};
			& \node (S) {\(s(t)\)};
			\\
		&&&& \coordinate (phiq);
			\\[-3mm]
		& \coordinate (vmq);
			& \node[twoportshape] (B2Lq) {};
			& \coordinate (mq);
			& \node[mixer] (Mq) {};
			&
			&
			& \coordinate (sq);
			\\
	};

	% Add missing lables
	\node at (H.center) {\large \(\mathcal{H}\)};
	\node at (B2Li.center) {\textsf{B2L}};
	\node at (B2Lq.center) {\textsf{B2L}};
	\node at (BSp) {\textsf{BSp}};

	% Add connections
	\begin{scope}[thick, -latex]
		\draw (M) -- (BSp.west);

		\draw (BSp.north) |- (B2Li.west);
		\draw (B2Li.east) -- (Mi.west);
		\draw (Mi.east) -| (SUM.north);

		\draw (BSp.south) |- (B2Lq.west);
		\draw (B2Lq.east) -- (Mq.west);
		\draw (Mq.east) -| (SUM.south);

		\draw (SUM.east) -- (S);

		\draw (OSC.east) -| (Mi.south);
		\draw (OSC.west) -- (H.east);
		\draw (H.south) -- (Mq.north);
	\end{scope}

	% Add signal labels
	\node[above right] at (vmi) {\(\vec{m}_i\)};
	\node[below right] at (vmq) {\(\vec{m}_q\)};

	\node[above] at (mi) {\(m_i(t)\)};
	\node[below] at (mq) {\(m_q(t)\)};

	\node[above right] at (phii) {\(\phi_i\)};
	\node[right, yshift = 1mm] at (phiq) {\(\phi_q\)};

	\node[above left] at (si) {\(s_i(t)\)};
	\node[below left] at (sq) {\(s_q(t)\)};

	% Draw digital signals
	\begin{scope}[font = \ttfamily\footnotesize, text = blue!70!white]
		\node[above = 1mm of M, xshift = 2mm] {\(\ldots 1100101\)};
		\node[above = 7mm of vmi, xshift = 3mm]
			{\(\overbracket[.8pt]{\,11\ldots 00\,}^{\sqrt{M} \text{ bits}}\)};
	\end{scope}

	% Draw analog waveform
	\begin{scope}[font = \ttfamily\tiny]
		\coordinate (O) at ($(mi)+(-2mm,10.5mm)$);

		\node[left, red!70!white, anchor = east, text width = 8mm, align = right]
			at ($(O) + (-2mm,0)$) {\(2^{\sqrt{M}}\) levels};

		\foreach \y in {-3mm,0,3mm} {
			\draw[gray, densely dotted] (O) ++(-2mm,\y) -- ++(22mm,0);
		}

		\draw[thick, draw = red!70!white] (O)
			-- ++(3mm,0) -- ++(0,-3mm) -- ++(3mm,0) -- ++(0,6mm)
			-- ++(3mm,0) -- ++(0,-3mm) -- ++(3mm,0) -- ++(0,-3mm)
			-- ++(3mm,0) -- ++(0,3mm)  -- ++(3mm,0);
	\end{scope}

	% Draw constellation diagram
	\begin{scope}
		\coordinate (O) at ($(S)+(-7mm,8mm)$);
		\draw[gray, -latex] (O) ++(-2mm,0) -- ++(12mm,0) node[right] {\tiny \(\phi_i\)};
		\draw[gray, -latex] (O) ++(0,-2mm) -- ++(0,12mm) node[above] {\tiny \(\phi_q\)};

		\node[
			circle, thick,
			minimum size = 3pt,
			inner sep = 0, outer sep = .8pt,
			draw = gray, fill = red!50!white
		] (P) at ($(O)+(5mm, 4mm)$) {};

		\node[gray, above right] at (P) {\tiny \(s\)};

		\draw[gray, densely dotted]
			(P) -- (P |- O)
			(P) -- (P -| O);
	\end{scope}


	% Background elements
	\begin{pgfonlayer}{background}
		\fill[left color = white, right color = blue!20, draw = white]
			($(B2Li.north) + (0,1.1)$) coordinate (D) rectangle ($(B2Lq.south) - (3,1)$);
		\fill[right color = white, left color = red!20, draw = white]
			($(B2Li.north) + (0,1.1)$) coordinate (A) rectangle ($(B2Lq.south) + (9,-1)$);

		\node[blue!50, anchor = south east, font = \ttfamily\bfseries, xshift = -4mm]
			at (D) {\bfseries\ttfamily Digital bits};
		\node[red!50, anchor = south west, font = \bfseries\ttfamily, xshift = 4mm]
			at (A) {Analog waveform};
	\end{pgfonlayer}
\end{circuitikz}

	\caption{
		%% TODO: caption
		\label{fig:quadrature-modulation}
	}
\end{figure}

Quadrature amplitude modulation is a family of modern digital modulation methods, that use an analog carrier signal. The simple yet effective idea behind QAM is to encode extra information into an orthogonal carrier signal, thus increasing the number of bits sent per unit of time. A diagram showing the process is found in figure \ref{fig:quadrature-modulation}.

%% TODO: Quick par on "we will dicusss M-Ary QAM"

\paragraph{Bit splitter}

As mentioned earlier, quadrature modulation allows sending more than one bit per unit time. The first step to do it is to use a so called bit splitter, converts the continuous data stream \(m(n)\) into pairs of chunks of \(\sqrt{M}\) bits. The two bit vectors of length \(\sqrt{M}\), denoted by \(\vec{m}_i\) and \(\vec{m}_q\), are called in-phase and quadrature component respectively. The reason will become more clear later.

\paragraph{Binary to level converter}

%% TODO: explain why gray code

Both bit vectors \(\vec{m}_i, \vec{m}_q \in \{0,1\}^{\sqrt{M}}\) are sent through a binary to level converter. It's purpose is to reinterpret the bit vector as a number, usually in gray code, and to convert them into an analog waveform, which we will denote with \(m_i(t)\) and \(m_q(t)\) respectively. Mathematically the binary to level converter can be described as:
\begin{equation}
	m_i(t) = \text{Level}(\vec{m}_i) \cdot p(t),
\end{equation}
i.e. with a pulse function \(p(t)\) (typically a root raised cosine to optimize for bandwidth) scaled by the interpreted binary value, which we will write here with a ``Level'' function. So at this point the analog waveform is already encoding \(\sqrt{M}\) bits per unit time, but actually it is possible to do better.

\paragraph{Mixer}

Having analog level signals, it is this now possible to mix them with radio frequency carriers. Because there are two waveforms, one might expect that two carrier frequencies are necessary, however this is not the case.

The two component \(m_i(t)\) and \(m_q(t)\) are mixed with two different periodic signals \(\phi_i(t)\) and \(\phi_q(t)\) that have the same frequency \(\omega_c = 2\pi / T\). Now the clever part: the carrier functions are picked to be \emph{orthonormal}, mathematically this is expressed by the conditions
\begin{subequations}
	\begin{align}
		\langle \phi_i | \phi_q \rangle
			&= \int_T \phi_i^* \phi_q \, dt = \int_T \phi_i \phi_q^* \, dt
			= 0, \text{ and } \\
		\langle \phi_k | \phi_k \rangle
			&= \int_T \phi_k^* \phi_k \,dt = 1,
			\text{ where } k \text{ can be either } i \text{ or } q.
	\end{align}
\end{subequations}

These rather abstract conditions remarkably allow for something very special. By defining a new signal 
\begin{equation}
	s = m_i\phi_i + m_q\phi_q,
\end{equation}
%% TODO: is this assumption correct?
notice that assuming \(m_i\) and \(m_q\) are constant over the period carrier's period \(T\),
\begin{align*}
	\langle s | \phi_i \rangle = \int_T s^* \phi_i \,dt
		&= \int m_i \phi_i^* \phi_i + m_q \phi_q^* \phi_i \,dt \\
		&= m_i \underbrace{\int_T \phi_i^* \phi_i \,dt}_{1}
			+ m_q \underbrace{\int_T \phi_q^* \phi_i \,dt}_{0} = m_i,
\end{align*}
which effectively means that it is possible to isolate a single component of the signal out of \(s\). The same of course works with \(\phi_q\) as well resulting in \(\langle s | \phi_q \rangle = m_q\).

% This formulation is rather abstract, in practice we usually pick \(\phi_i(t) = \cos(\omega_c t)\) and \(\phi_q(t) = j\sin(\omega_c t)\). 

% \begin{figure}
% 	\centering
% 	% vim: set ts=2 sw=2 noet spell:

\begin{tikzpicture}
	\begin{axis}[
		axis lines = middle,
		colormap/cool,
	]
		\pgfmathsetmacro{\fc}{10}

		\addplot3[]
			{sin(x) + 1};
	\end{axis}
\end{tikzpicture}

% 	\caption{
% 		% TODO: write caption
% 		\label{fig:qpks-constellation}
% 	}
% \end{figure}

\subsection{Phase Shift Keying (PSK)}

PSK is a popular modulation type for data transmission\cite{Meyer2011}. With a bipolar binary signal, the amplitude remains constant and only the phase will be changed with phase jumps of 180 degrees, which can be seen as a multiplication of the carrier signal with $\pm$ 1. That is alow known as binary phase shift keying.

% \begin{figure}
% 	% TODO: Better Image
% 	% https://sites.google.com/site/billmahroukelec675/bipolar-phase-shift-keying
% 	\includegraphics[width=5cm]{./image/BPSK2.png}
% \end{figure}

\subsection{Quadrature Phase Shift Keying (QPSK)}

Two bits are modulated at ones with the same bandwidth as a 2-PSK so more informations are transmitted at the same time. \cite{Meyer2011}
%TODO: Image Signal Raum 
Most times there is noise and the points on the constellation diagram become a surface. 
If the surfaces overlap there will be a problem with decoding. 

\section{Fading}

\subsection{Geometric Model}


\subsection{Statistical Model}

%% TODO: write about advantage of statistical model instead of geometric

\paragraph{Continuous time model}

Continuous time small scale fading channel response.

time varying channel impulse response:
\begin{equation}
	h(t, \tau) = \sum_k c_k (t) \delta(\tau - \tau_k(t))
\end{equation}

received signal \(y = h * x\), i.e. convolution with channel model. 

\subsection{Time discretization of the model}

%% TODO: explain why

Assume \(x\) is a time discrete signal with and bandwidth \(W\), thus the pulse is sinc shaped
\begin{equation}
	x(t) = \sum_n x[n] \sinc(t/T - n)
\end{equation}
Ideal sampling at rate \(2W\) of \(y\) gives

	% vim: set ts=2 sw=2 noet:

\chapter{Implementation}

\section{Simulaton}
%%TO DO: quelle https://wiki.gnuradio.org/

For the simulation task and after for the Hardware part, the open-source Software GNU Radio has been chosen. This software uses toolboxes for signal processing systems too simulate or/and implement a software-defined radio, based on Python and some C++ implementations for some rapid-application-development environments. The toolboxes can simply, with the help of the graphical user interface, used by drag-and-drop. The Boxes are used to write applications, to receive or to transmit date for a digital system. Some blocks like different filters, channel codes or demodulator elements and a lot more are already implemented. For missing application new elements can be added by coding own block. With the help of the GNU Radio software those toolboxes can easily get connected to each other, creating data streams. 


\subsection{16QAM Simulation}

\paragraph{Source}

\paragraph{Modulator}

\paragraph{Channel Mode}

\paragraph{Polyphase Clock Sync}

\paragraph{Equalizer}

\paragraph{Costas Loop}

\paragraph{Decoder}




\subsection{Simulation Fading}
%% TO DO: Quelle: 

\subsubsection{FIR-Filter}

For a first simple Simulation of the Fading effect. A FIR-Filter has been integrate in the Simulation Model.
\begin{figure}

 	\includegraphics[width=10cm]{./figures/screenshots/QAM16_Fading_2.png}
 \end{figure}


\section{Hardware}

As Hardware we chosen the USRP B210 from Ettus Research, with the following specifications shown in Tab. \ref{tab:USRP B210 specifications}. Because this SDR is more than enough for our requires.

\subsection{Hardware setup}

The First Hardware set up was from the first SDR to the second, with a coaxial cable in between. 

The second was with the antennas. 2.4GHz. 

% To Do: Picture



\begin{table}[]
	%To DO sepzifikationen ampssen / genauer? https://www.ettus.com/wp-content/uploads/2019/01/b200-b210_spec_sheet.pdf
	%https://kb.ettus.com/B200/B210/B200mini/B205mini#FAQ
	\caption{USRP B210 specifications}
	\begin{tabular}[h]{|c|c|}
		\hline
		Dimensions & 9.7 x 15.5 x 1.5 cm \\
		\hline
		Ports &
2 TX , 2 RX, Half  or Full Duplex\\
		\hline
		RF frequencies & from 70MHz to 6GHz\\
		\hline
		Bandwidth & 200kHz-56MHz\\
		\hline 
		External reference input & 10 MHz \\
		\hline
	\end{tabular}
\label{tab:USRP B210 specifications}
\end{table}


\section{Measurements}



\section{Results}

	% vim: set ts=2 sw=2 noet:

\chapter{Conclusions}

The goal to build a functional demonstrator could be realized, unfortunately not with all futures as originally planned. A functional receiver and transmitter chain, for QPSK were implemented,
but the 16-QAM dosent work as hoped.
Some different typ of multiple fading model were tested and illustrated.
Two different Models for the simulation options are build. One discrete time model whish is basicly a FIR filter in the channel, the other with a statistical model which is based on a GR block.
And another file to implement the hardware with. Unfortunately it wasent possible to measure those models in a meaningful way. For that a least square approximation could be used as described in the further steps. An other difficulty is to reproduce the same effect in a simulation compare with the hardware, because of al the side effect of an environment, which cant be predicted in a simulation.




\section{Further Steps}

To improve this project a simulation environment could be implementer, which can be replicated for more accurate measurements. So then it is possible to compare the simulation with the measurements.

An other  developnet basde on this project could be to show the BER with the help of a least Square approximation.(like in the paper)

	%% TODO: remove in final version
	\printskelnotes
	\printbibliography

\end{document}
