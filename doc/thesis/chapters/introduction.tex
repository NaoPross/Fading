% vim: set ts=2 sw=2 noet spell spelllang=en:

\chapter{Introduction}

\section{Background}

It is undeniable that in the last two decades modern wireless devices have become extremely ubiquitous, and are no longer employed under carefully chosen conditions.

Nowadays smart phones and internet of things (IoT) devices are carried around by everyone and have to work in environments that are very far from ideal. Furthermore, next generation wireless devices in urban environments will also include vehicles (cars, buses and trains)\cite{AntonescuTB17}, where reliability of intra-vehicular communication directly translates into safety. Whereas in rural regions, developing countries as well as other low-user density areas wireless transmission links using mesh networks have become a practical alternative to wired broadband\cite{Macmillan2019tidal,Subramanian2006rethinking,Flickenger2007wireless}.

All of the mentioned cases have a common problem caused by \emph{the fading effect}, which degrade the reliability of the link\cite{Mathis}. This was foreseen\cite{Frederiksen2002overview,Maddocks1993introduction} and today most modern transmission schemes implement measures to reduce the effects fading\cite{Mathis,Hsu}.

\section{Task description}

As described in the document given at the beginning of the semester:

\begin{quote}
	The goal is to develop a SDR-based demonstrator, consisting of one transmitter and one receiver, to illustrate the impact of different fading effects on the signal. To get a brief understanding of the concept of fading channels, the project should be started with a literature research followed by simulation of different scenarios, which then can be reproduced by measurements.
\end{quote}

\section{Overview}
