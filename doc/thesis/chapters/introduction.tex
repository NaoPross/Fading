% vim: set ts=2 sw=2 noet spell spelllang=en:

\chapter{Introduction}

\section{Background}

It is undeniable that in the last two decades modern wireless devices have become extremely ubiquitous, and are no longer employed under carefully chosen conditions.

Nowadays smart phones and internet of things (IoT) devices and many other wireless devices are carried around by everyone and have to work in environments that are very far from ideal. Furthermore in addition to the already large class of networked appliances, next generation wireless devices in urban environments will include the new category of vehicles \cite{AntonescuTB17}, where reliability of intra-vehicular communication directly translates into safety. While at the same time in rural regions, developing countries as well as other low-user density areas wireless transmission links using mesh networks have become a practical alternative to wired broadband \cite{Macmillan2019tidal,Subramanian2006rethinking,Flickenger2007wireless}.

The study of problems concerning wireless devices is thus a very relevant topic today. More specifically, a common issue in the previously mentioned use cases is the so called \emph{multipath fading effect}, which degrade the reliability of the transmission link \cite{Mathis, Gallager}. The presence of fading was actually foreseen \cite{Frederiksen2002overview,Maddocks1993introduction} and today most modern transmission schemes implement measures to reduce the effects fading \cite{Mathis,Hsu}.

This work studies the multipath fading effect, and how it affects modern digital transmission systems that use quadrature amplitude (QAM) and phase shift keying (PSK) modulation.

\section{Task description}

As described in the document given at the beginning of the semester:
\begin{quote}
	The goal is to develop a SDR-based demonstrator, consisting of one transmitter and one receiver, to illustrate the impact of different fading effects on the signal. To get a brief understanding of the concept of fading channels, the project should be started with a literature research followed by simulation of different scenarios, which then can be reproduced by measurements.
\end{quote}
The task description document is found in the appendix.

\section{Overview}

\skelpar{Overview of the whole document.}
