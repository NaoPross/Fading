% vim: set ts=2 sw=2 noet spell:

\chapter{Theory}

\section{Review of modulation schemes}

\begin{figure}
	\centering
	% vim: set ts=2 sw=2 noet spell:

\begin{tikzpicture}[
		% show background rectangle,
		box/.style = {
			font = \small\sffamily\bfseries,
			draw, thick, fill = white,
			minimum height = 10mm,
			minimum width = 25mm,
		},
	]
	\matrix[
		nodes = {box}, column sep = 10mm, row sep = 10mm,
	]{
		\node (SRC) {Source}; &
		\node (ENC) {Encoder}; &
		\node (MOD) {Modulator}; \\
		&& \node (CHN) {Channel}; \\
		\node (SNK) {Sink}; &
		\node (DEC) {Decoder}; &
		\node (DMD) {Demodulator}; \\
	};
	\draw[very thick, -latex]
		(SRC) edge node[midway, above] {\(m\)}    (ENC)
		(ENC) edge node[midway, above] {\(m_e\)}  (MOD)
		(MOD) edge node[midway, right] {\(s\)}    (CHN)
		(CHN) edge node[midway, right] {\(y\)}    (DMD)
		(DMD) edge node[midway, above] {\(m_e'\)} (DEC)
		(DEC) edge node[midway, above] {\(m'\)}   (SNK)
	;

	\draw[very thick, -latex]
		($(MOD.east)+(5mm,0)$) node[right] (CARR) {Carrier \(e^{j(\omega_c t + \phi)}\)} to (MOD);

	\draw[very thick, -latex]
		($(CHN.east)+(5mm,0)$) node[right] (NOISE) {Noise \(n\)} to (CHN);

	% \draw[very thick, -latex]
	% 	($(CHN.west)-(5mm,0)$) node[left] (FADE) {Fading} to (CHN);

	\draw[very thick, -latex]
		($(DMD.east)+(5mm,0)$) node[right] (SYNC) {Sync} to (DMD);

	\begin{pgfonlayer}{background}
		\fill[lightgray!20] ($(SRC.north west)+(-5mm,5mm)$) node (TX) {}
			rectangle ($(MOD.south -| CARR.east)+(5mm,-5mm)$);

		\fill[lightgray!20] ($(SNK.north west)+(-5mm,5mm)$) node (RX) {}
			rectangle ($(DMD.south -| CARR.east)+(5mm,-5mm)$);

		\node[gray, font = \sffamily\bfseries, anchor = south west]
			at (TX.north) {Transmitter};

		\node[gray, font = \sffamily\bfseries, anchor = south west]
			at (RX.north) {Receiver};
	\end{pgfonlayer}
\end{tikzpicture}

	\caption{
		Block diagram for a general wireless communication system with annotated signal names.
		Frequency domain representations of signals use the uppercase symbol of their respective time domain name.
		\label{fig:notation}
	}
\end{figure}

In this section we will briefly give the mathematical background required by the modulation schemes used in the project. For conciseness encoding schemes and (digital) signal processing calculations are left out and discussed later. Thus for this section \(m_e = m\).

\paragraph{AM / DSB}

Ordinary amplitude modulation (AM), sometimes also known as double sideband (DSB) modulation in its simplest form is mathematically formulated in time and frequency domain through the following equations\cite{Hsu}:
\begin{subequations}
	\begin{align}
		s(t) &= \big( 1 + \mu m(t) \big) \cdot \Re{e^{j\omega_c t}}, \\
		S(\omega) &= \pi\delta(-\omega_c)
			+ \pi\delta(\omega_c)
			+ \frac{\mu}{2} M(\omega - \omega_c)
			+ \frac{\mu}{2} M(\omega + \omega_c).
	\end{align}
\end{subequations}
Where \(\mu > 0\) is the so called modulation factor, that can be adjusted to avoid clipping and improve performance.

\subsection{Quadrature amplitude modulation (QAM)}

Quadrature amplitude modulation is a family of modern digital modulation methods, that use an analog carrier signal. The simple yet effective idea behind QAM is to encode extra information into an orthogonal carrier signal, thus increasing the number of bits sent per unit of time.

\paragraph{Mathematical formulation}

For QAM we wish to split the signal space into two orthonormal basis functions \(\psi_i\) and \(\psi_q\), such that the inner product \(\langle \psi_i | \psi_q \rangle = 0\). The two functions \(\psi_i\) and \(\psi_q\) are called in-phase and quadrature component. For a cosinusoidal in-phase carrier component we obtain from the previous requirement that \(\psi_i = \sqrt{\omega_c} \cos(\omega_c t), \text{ and } \psi_q = \sqrt{\omega_c} \sin(\omega_c t)\)

Now, let \(\vec{m} \in \{0,1\}^n\) be a binary row vector that encodes our message.

\section{Problem description}

\section{Geometric Model}

\section{Statistical Model}

%% TODO: write about advantage of statistical model instead of geometric
%% TODO: review and rewrite notes

\subsection{Continuous time model}

Continuous time small scale fading channel response. \cite{Alimohammad2009}

time varying channel impulse response:
\begin{equation}
	h(t, \tau) = \sum_k c_k (t) \delta(\tau - \tau_k(t))
\end{equation}

received signal \(y = h * x\), i.e. convolution with channel model. 

\subsection{Time discretization of the model}

%% TODO: explain why

Assume \(x\) is a time discrete signal with and bandwidth \(W\), thus the pulse is sinc shaped
\begin{equation}
	x(t) = \sum_n x[n] \sinc(t/T - n)
\end{equation}
Ideal sampling at rate \(2W\) of \(y\) gives
