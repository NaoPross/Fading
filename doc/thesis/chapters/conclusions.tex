% vim: set ts=2 sw=2 noet:

\chapter{Conclusions}

The goal to build a functional demonstrator could be realized, unfortunately not with all futures as originally planned. A functional receiver and transmitter chain, for QPSK were implemented,
but the 16-QAM dosent work as hoped.
Some different typ of multiple fading model were tested and illustrated.
Two different Models for the simulation options are build. One discrete time model whish is basicly a FIR filter in the channel, the other with a statistical model which is based on a GR block.
And another file to implement the hardware with. Unfortunately it wasent possible to measure those models in a meaningful way. For that a least square approximation could be used as described in the further steps. An other difficulty is to reproduce the same effect in a simulation compare with the hardware, because of al the side effect of an environment, which cant be predicted in a simulation.




\section{Further Steps}

To improve this project a simulation environment could be implementer, which can be replicated for more accurate measurements. So then it is possible to compare the simulation with the measurements.

An other  developnet basde on this project could be to show the BER with the help of a least Square approximation.(like in the paper)