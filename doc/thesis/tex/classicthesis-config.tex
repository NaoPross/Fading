% ****************************************************************************************************
% classicthesis-config.tex
% formerly known as loadpackages.sty, classicthesis-ldpkg.sty, and classicthesis-preamble.sty
% Use it at the beginning of your ClassicThesis.tex, or as a LaTeX Preamble
% in your ClassicThesis.{tex,lyx} with % ****************************************************************************************************
% classicthesis-config.tex
% formerly known as loadpackages.sty, classicthesis-ldpkg.sty, and classicthesis-preamble.sty
% Use it at the beginning of your ClassicThesis.tex, or as a LaTeX Preamble
% in your ClassicThesis.{tex,lyx} with % ****************************************************************************************************
% classicthesis-config.tex
% formerly known as loadpackages.sty, classicthesis-ldpkg.sty, and classicthesis-preamble.sty
% Use it at the beginning of your ClassicThesis.tex, or as a LaTeX Preamble
% in your ClassicThesis.{tex,lyx} with % ****************************************************************************************************
% classicthesis-config.tex
% formerly known as loadpackages.sty, classicthesis-ldpkg.sty, and classicthesis-preamble.sty
% Use it at the beginning of your ClassicThesis.tex, or as a LaTeX Preamble
% in your ClassicThesis.{tex,lyx} with \input{classicthesis-config}

% If you like the classicthesis, then I would appreciate a postcard.
% My address can be found in the file ClassicThesis.pdf. A collection
% of the postcards I received so far is available online at
% http://postcards.miede.de


% ****************************************************************************************************
% 0. Set the encoding of your files. UTF-8 is the only sensible encoding nowadays. If you can't read
% äöüßáéçèê∂åëæƒÏ€ then change the encoding setting in your editor, not the line below. If your editor
% does not support utf8 use another editor!
\PassOptionsToPackage{utf8}{inputenc}
  \usepackage{inputenc}

\PassOptionsToPackage{T1}{fontenc} % T2A for cyrillics
  \usepackage{fontenc}


% ****************************************************************************************************
% 1. Configure classicthesis for your needs here, e.g., remove "drafting" below
% in order to deactivate the time-stamp on the pages
% (see ClassicThesis.pdf for more information):

\PassOptionsToPackage{
  drafting=true,    % print version information on the bottom of the pages
  tocaligned=false, % the left column of the toc will be aligned (no indentation)
  dottedtoc=false,  % page numbers in ToC flushed right
  eulerchapternumbers=true, % use AMS Euler for chapter font (otherwise Palatino)
  linedheaders=false,       % chaper headers will have line above and beneath
  floatperchapter=true,     % numbering per chapter for all floats (i.e., Figure 1.1)
  eulermath=false,  % use awesome Euler fonts for mathematical formulae (only with pdfLaTeX)
  beramono=true,    % toggle a nice monospaced font (w/ bold)
  palatino=true,    % deactivate standard font for loading another one, see the last section at the end of this file for suggestions
  style=classicthesis % classicthesis, arsclassica
}{classicthesis}


% ****************************************************************************************************
% 2. Personal data and user ad-hoc commands (insert your own data here)

\newcommand{\myTitle}{Fading Illustration with Software Defined Radio\xspace}
\newcommand{\mySubtitle}{\xspace}
\newcommand{\myDegree}{\xspace}
\newcommand{\myName}{Halter Sara Cinzia \& Pross Naoki Sean\xspace}
\newcommand{\myProf}{Prof. Dr. Heinz Mathis\xspace}
\newcommand{\myOtherProf}{\xspace}
\newcommand{\mySupervisor}{Michel Nyffenegger\xspace}
\newcommand{\myFaculty}{University of Applied Sciences of East Switzerland\xspace}
\newcommand{\myDepartment}{Institute for Communication Systems ICOM\xspace}
\newcommand{\myUni}{OST Fachhochschule\xspace}
\newcommand{\myLocation}{Rapperswil\xspace}
\newcommand{\myTime}{December 2021\xspace}
\newcommand{\myVersion}{\classicthesis}

% Setup, finetuning, and useful commands
\newcommand{\ie}{i.\,e.}
\newcommand{\Ie}{I.\,e.}
\newcommand{\eg}{e.\,g.}
\newcommand{\Eg}{E.\,g.}

% ****************************************************************************************************
% 3. Loading some handy packages

% ********************************************************************
% Packages with options that might require adjustments

\PassOptionsToPackage{american}{babel} % change this to your language(s), main language last
    \usepackage{babel}

\usepackage{csquotes}
\PassOptionsToPackage{%
  %backend=biber,bibencoding=utf8, %instead of bibtex
  backend=bibtex8,bibencoding=ascii,%
  language=auto,%
  style=numeric-comp,%
  %style=authoryear-comp, % Author 1999, 2010
  %bibstyle=authoryear,dashed=false, % dashed: substitute rep. author with ---
  sorting=nyt, % name, year, title
  maxbibnames=10, % default: 3, et al.
  %backref=true,%
  natbib=true % natbib compatibility mode (\citep and \citet still work)
}{biblatex}
    \usepackage{biblatex}

\PassOptionsToPackage{fleqn}{amsmath}       % math environments and more by the AMS
  \usepackage{amsmath}

% ********************************************************************
% General useful packages

\usepackage{graphicx} %
\usepackage{scrhack} % fix warnings when using KOMA with listings package
\usepackage{xspace} % to get the spacing after macros right
\PassOptionsToPackage{printonlyused,smaller}{acronym}
  \usepackage{acronym} % nice macros for handling all acronyms in the thesis
  %\renewcommand{\bflabel}[1]{{#1}\hfill} % fix the list of acronyms --> no longer working
  %\renewcommand*{\acsfont}[1]{\textsc{#1}}
  %\renewcommand*{\aclabelfont}[1]{\acsfont{#1}}
  %\def\bflabel#1{{#1\hfill}}
  \def\bflabel#1{{\acsfont{#1}\hfill}}
  \def\aclabelfont#1{\acsfont{#1}}
  
% ****************************************************************************************************
% 4. Setup floats: tables, (sub)figures, and captions

\usepackage{tabularx} % better tables
  \setlength{\extrarowheight}{3pt} % increase table row height
\newcommand{\tableheadline}[1]{\multicolumn{1}{l}{\spacedlowsmallcaps{#1}}}
\newcommand{\myfloatalign}{\centering} % to be used with each float for alignment
\usepackage{subfig}

% ****************************************************************************************************
% 5. Setup code listings

\usepackage{listings}
%\lstset{emph={trueIndex,root},emphstyle=\color{BlueViolet}}%\underbar} % for special keywords
\lstset{language=[LaTeX]Tex,%C++,
  morekeywords={PassOptionsToPackage,selectlanguage},
  keywordstyle=\color{RoyalBlue},%\bfseries,
  basicstyle=\small\ttfamily,
  %identifierstyle=\color{NavyBlue},
  commentstyle=\color{Green}\ttfamily,
  stringstyle=\rmfamily,
  numbers=none,%left,%
  numberstyle=\scriptsize,%\tiny
  stepnumber=5,
  numbersep=8pt,
  showstringspaces=false,
  breaklines=true,
  %frameround=ftff,
  %frame=single,
  belowcaptionskip=.75\baselineskip
  %frame=L
}



% ********************************************************************
% Fine-tune hyperreferences (hyperref should be called last)

\usepackage{hyperref}
\hypersetup{%
  %draft, % hyperref's draft mode, for printing see below
  colorlinks=true, linktocpage=true, pdfstartpage=3, pdfstartview=FitV,%
  % uncomment the following line if you want to have black links (e.g., for printing)
  %colorlinks=false, linktocpage=false, pdfstartpage=3, pdfstartview=FitV, pdfborder={0 0 0},%
  breaklinks=true, pageanchor=true,%
  pdfpagemode=UseNone, %
  % pdfpagemode=UseOutlines,%
  plainpages=false, bookmarksnumbered, bookmarksopen=true, bookmarksopenlevel=1,%
  hypertexnames=true, pdfhighlight=/O,%nesting=true,%frenchlinks,%
  urlcolor=CTurl, linkcolor=CTlink, citecolor=CTcitation, %pagecolor=RoyalBlue,%
  %urlcolor=Black, linkcolor=Black, citecolor=Black, %pagecolor=Black,%
  pdftitle={\myTitle},%
  pdfauthor={\textcopyright\ \myName, \myUni, \myFaculty},%
  pdfsubject={},%
  pdfkeywords={},%
  pdfcreator={pdfLaTeX},%
  pdfproducer={LaTeX with hyperref and classicthesis}%
}


% ********************************************************************
% Setup autoreferences (hyperref and babel)

% There are some issues regarding autorefnames
% http://www.tex.ac.uk/cgi-bin/texfaq2html?label=latexwords
% you have to redefine the macros for the
% language you use, e.g., american, ngerman
% (as chosen when loading babel/AtBeginDocument)
\makeatletter
\@ifpackageloaded{babel}%
  {%
    \addto\extrasamerican{%
      \renewcommand*{\figureautorefname}{Figure}%
      \renewcommand*{\tableautorefname}{Table}%
      \renewcommand*{\partautorefname}{Part}%
      \renewcommand*{\chapterautorefname}{Chapter}%
      \renewcommand*{\sectionautorefname}{Section}%
      \renewcommand*{\subsectionautorefname}{Section}%
      \renewcommand*{\subsubsectionautorefname}{Section}%
    }%
    \addto\extrasngerman{%
      \renewcommand*{\paragraphautorefname}{Absatz}%
      \renewcommand*{\subparagraphautorefname}{Unterabsatz}%
      \renewcommand*{\footnoteautorefname}{Fu\"snote}%
      \renewcommand*{\FancyVerbLineautorefname}{Zeile}%
      \renewcommand*{\theoremautorefname}{Theorem}%
      \renewcommand*{\appendixautorefname}{Anhang}%
      \renewcommand*{\equationautorefname}{Gleichung}%
      \renewcommand*{\itemautorefname}{Punkt}%
    }%
      % Fix to getting autorefs for subfigures right (thanks to Belinda Vogt for changing the definition)
      \providecommand{\subfigureautorefname}{\figureautorefname}%
    }{\relax}
\makeatother


% If you like the classicthesis, then I would appreciate a postcard.
% My address can be found in the file ClassicThesis.pdf. A collection
% of the postcards I received so far is available online at
% http://postcards.miede.de


% ****************************************************************************************************
% 0. Set the encoding of your files. UTF-8 is the only sensible encoding nowadays. If you can't read
% äöüßáéçèê∂åëæƒÏ€ then change the encoding setting in your editor, not the line below. If your editor
% does not support utf8 use another editor!
\PassOptionsToPackage{utf8}{inputenc}
  \usepackage{inputenc}

\PassOptionsToPackage{T1}{fontenc} % T2A for cyrillics
  \usepackage{fontenc}


% ****************************************************************************************************
% 1. Configure classicthesis for your needs here, e.g., remove "drafting" below
% in order to deactivate the time-stamp on the pages
% (see ClassicThesis.pdf for more information):

\PassOptionsToPackage{
  drafting=true,    % print version information on the bottom of the pages
  tocaligned=false, % the left column of the toc will be aligned (no indentation)
  dottedtoc=false,  % page numbers in ToC flushed right
  eulerchapternumbers=true, % use AMS Euler for chapter font (otherwise Palatino)
  linedheaders=false,       % chaper headers will have line above and beneath
  floatperchapter=true,     % numbering per chapter for all floats (i.e., Figure 1.1)
  eulermath=false,  % use awesome Euler fonts for mathematical formulae (only with pdfLaTeX)
  beramono=true,    % toggle a nice monospaced font (w/ bold)
  palatino=true,    % deactivate standard font for loading another one, see the last section at the end of this file for suggestions
  style=classicthesis % classicthesis, arsclassica
}{classicthesis}


% ****************************************************************************************************
% 2. Personal data and user ad-hoc commands (insert your own data here)

\newcommand{\myTitle}{Fading Illustration with Software Defined Radio\xspace}
\newcommand{\mySubtitle}{\xspace}
\newcommand{\myDegree}{\xspace}
\newcommand{\myName}{Halter Sara Cinzia \& Pross Naoki Sean\xspace}
\newcommand{\myProf}{Prof. Dr. Heinz Mathis\xspace}
\newcommand{\myOtherProf}{\xspace}
\newcommand{\mySupervisor}{Michel Nyffenegger\xspace}
\newcommand{\myFaculty}{University of Applied Sciences of East Switzerland\xspace}
\newcommand{\myDepartment}{Institute for Communication Systems ICOM\xspace}
\newcommand{\myUni}{OST Fachhochschule\xspace}
\newcommand{\myLocation}{Rapperswil\xspace}
\newcommand{\myTime}{December 2021\xspace}
\newcommand{\myVersion}{\classicthesis}

% Setup, finetuning, and useful commands
\newcommand{\ie}{i.\,e.}
\newcommand{\Ie}{I.\,e.}
\newcommand{\eg}{e.\,g.}
\newcommand{\Eg}{E.\,g.}

% ****************************************************************************************************
% 3. Loading some handy packages

% ********************************************************************
% Packages with options that might require adjustments

\PassOptionsToPackage{american}{babel} % change this to your language(s), main language last
    \usepackage{babel}

\usepackage{csquotes}
\PassOptionsToPackage{%
  %backend=biber,bibencoding=utf8, %instead of bibtex
  backend=bibtex8,bibencoding=ascii,%
  language=auto,%
  style=numeric-comp,%
  %style=authoryear-comp, % Author 1999, 2010
  %bibstyle=authoryear,dashed=false, % dashed: substitute rep. author with ---
  sorting=nyt, % name, year, title
  maxbibnames=10, % default: 3, et al.
  %backref=true,%
  natbib=true % natbib compatibility mode (\citep and \citet still work)
}{biblatex}
    \usepackage{biblatex}

\PassOptionsToPackage{fleqn}{amsmath}       % math environments and more by the AMS
  \usepackage{amsmath}

% ********************************************************************
% General useful packages

\usepackage{graphicx} %
\usepackage{scrhack} % fix warnings when using KOMA with listings package
\usepackage{xspace} % to get the spacing after macros right
\PassOptionsToPackage{printonlyused,smaller}{acronym}
  \usepackage{acronym} % nice macros for handling all acronyms in the thesis
  %\renewcommand{\bflabel}[1]{{#1}\hfill} % fix the list of acronyms --> no longer working
  %\renewcommand*{\acsfont}[1]{\textsc{#1}}
  %\renewcommand*{\aclabelfont}[1]{\acsfont{#1}}
  %\def\bflabel#1{{#1\hfill}}
  \def\bflabel#1{{\acsfont{#1}\hfill}}
  \def\aclabelfont#1{\acsfont{#1}}
  
% ****************************************************************************************************
% 4. Setup floats: tables, (sub)figures, and captions

\usepackage{tabularx} % better tables
  \setlength{\extrarowheight}{3pt} % increase table row height
\newcommand{\tableheadline}[1]{\multicolumn{1}{l}{\spacedlowsmallcaps{#1}}}
\newcommand{\myfloatalign}{\centering} % to be used with each float for alignment
\usepackage{subfig}

% ****************************************************************************************************
% 5. Setup code listings

\usepackage{listings}
%\lstset{emph={trueIndex,root},emphstyle=\color{BlueViolet}}%\underbar} % for special keywords
\lstset{language=[LaTeX]Tex,%C++,
  morekeywords={PassOptionsToPackage,selectlanguage},
  keywordstyle=\color{RoyalBlue},%\bfseries,
  basicstyle=\small\ttfamily,
  %identifierstyle=\color{NavyBlue},
  commentstyle=\color{Green}\ttfamily,
  stringstyle=\rmfamily,
  numbers=none,%left,%
  numberstyle=\scriptsize,%\tiny
  stepnumber=5,
  numbersep=8pt,
  showstringspaces=false,
  breaklines=true,
  %frameround=ftff,
  %frame=single,
  belowcaptionskip=.75\baselineskip
  %frame=L
}



% ********************************************************************
% Fine-tune hyperreferences (hyperref should be called last)

\usepackage{hyperref}
\hypersetup{%
  %draft, % hyperref's draft mode, for printing see below
  colorlinks=true, linktocpage=true, pdfstartpage=3, pdfstartview=FitV,%
  % uncomment the following line if you want to have black links (e.g., for printing)
  %colorlinks=false, linktocpage=false, pdfstartpage=3, pdfstartview=FitV, pdfborder={0 0 0},%
  breaklinks=true, pageanchor=true,%
  pdfpagemode=UseNone, %
  % pdfpagemode=UseOutlines,%
  plainpages=false, bookmarksnumbered, bookmarksopen=true, bookmarksopenlevel=1,%
  hypertexnames=true, pdfhighlight=/O,%nesting=true,%frenchlinks,%
  urlcolor=CTurl, linkcolor=CTlink, citecolor=CTcitation, %pagecolor=RoyalBlue,%
  %urlcolor=Black, linkcolor=Black, citecolor=Black, %pagecolor=Black,%
  pdftitle={\myTitle},%
  pdfauthor={\textcopyright\ \myName, \myUni, \myFaculty},%
  pdfsubject={},%
  pdfkeywords={},%
  pdfcreator={pdfLaTeX},%
  pdfproducer={LaTeX with hyperref and classicthesis}%
}


% ********************************************************************
% Setup autoreferences (hyperref and babel)

% There are some issues regarding autorefnames
% http://www.tex.ac.uk/cgi-bin/texfaq2html?label=latexwords
% you have to redefine the macros for the
% language you use, e.g., american, ngerman
% (as chosen when loading babel/AtBeginDocument)
\makeatletter
\@ifpackageloaded{babel}%
  {%
    \addto\extrasamerican{%
      \renewcommand*{\figureautorefname}{Figure}%
      \renewcommand*{\tableautorefname}{Table}%
      \renewcommand*{\partautorefname}{Part}%
      \renewcommand*{\chapterautorefname}{Chapter}%
      \renewcommand*{\sectionautorefname}{Section}%
      \renewcommand*{\subsectionautorefname}{Section}%
      \renewcommand*{\subsubsectionautorefname}{Section}%
    }%
    \addto\extrasngerman{%
      \renewcommand*{\paragraphautorefname}{Absatz}%
      \renewcommand*{\subparagraphautorefname}{Unterabsatz}%
      \renewcommand*{\footnoteautorefname}{Fu\"snote}%
      \renewcommand*{\FancyVerbLineautorefname}{Zeile}%
      \renewcommand*{\theoremautorefname}{Theorem}%
      \renewcommand*{\appendixautorefname}{Anhang}%
      \renewcommand*{\equationautorefname}{Gleichung}%
      \renewcommand*{\itemautorefname}{Punkt}%
    }%
      % Fix to getting autorefs for subfigures right (thanks to Belinda Vogt for changing the definition)
      \providecommand{\subfigureautorefname}{\figureautorefname}%
    }{\relax}
\makeatother


% If you like the classicthesis, then I would appreciate a postcard.
% My address can be found in the file ClassicThesis.pdf. A collection
% of the postcards I received so far is available online at
% http://postcards.miede.de


% ****************************************************************************************************
% 0. Set the encoding of your files. UTF-8 is the only sensible encoding nowadays. If you can't read
% äöüßáéçèê∂åëæƒÏ€ then change the encoding setting in your editor, not the line below. If your editor
% does not support utf8 use another editor!
\PassOptionsToPackage{utf8}{inputenc}
  \usepackage{inputenc}

\PassOptionsToPackage{T1}{fontenc} % T2A for cyrillics
  \usepackage{fontenc}


% ****************************************************************************************************
% 1. Configure classicthesis for your needs here, e.g., remove "drafting" below
% in order to deactivate the time-stamp on the pages
% (see ClassicThesis.pdf for more information):

\PassOptionsToPackage{
  drafting=true,    % print version information on the bottom of the pages
  tocaligned=false, % the left column of the toc will be aligned (no indentation)
  dottedtoc=false,  % page numbers in ToC flushed right
  eulerchapternumbers=true, % use AMS Euler for chapter font (otherwise Palatino)
  linedheaders=false,       % chaper headers will have line above and beneath
  floatperchapter=true,     % numbering per chapter for all floats (i.e., Figure 1.1)
  eulermath=false,  % use awesome Euler fonts for mathematical formulae (only with pdfLaTeX)
  beramono=true,    % toggle a nice monospaced font (w/ bold)
  palatino=true,    % deactivate standard font for loading another one, see the last section at the end of this file for suggestions
  style=classicthesis % classicthesis, arsclassica
}{classicthesis}


% ****************************************************************************************************
% 2. Personal data and user ad-hoc commands (insert your own data here)

\newcommand{\myTitle}{Fading Illustration with Software Defined Radio\xspace}
\newcommand{\mySubtitle}{\xspace}
\newcommand{\myDegree}{\xspace}
\newcommand{\myName}{Halter Sara Cinzia \& Pross Naoki Sean\xspace}
\newcommand{\myProf}{Prof. Dr. Heinz Mathis\xspace}
\newcommand{\myOtherProf}{\xspace}
\newcommand{\mySupervisor}{Michel Nyffenegger\xspace}
\newcommand{\myFaculty}{University of Applied Sciences of East Switzerland\xspace}
\newcommand{\myDepartment}{Institute for Communication Systems ICOM\xspace}
\newcommand{\myUni}{OST Fachhochschule\xspace}
\newcommand{\myLocation}{Rapperswil\xspace}
\newcommand{\myTime}{December 2021\xspace}
\newcommand{\myVersion}{\classicthesis}

% Setup, finetuning, and useful commands
\newcommand{\ie}{i.\,e.}
\newcommand{\Ie}{I.\,e.}
\newcommand{\eg}{e.\,g.}
\newcommand{\Eg}{E.\,g.}

% ****************************************************************************************************
% 3. Loading some handy packages

% ********************************************************************
% Packages with options that might require adjustments

\PassOptionsToPackage{american}{babel} % change this to your language(s), main language last
    \usepackage{babel}

\usepackage{csquotes}
\PassOptionsToPackage{%
  %backend=biber,bibencoding=utf8, %instead of bibtex
  backend=bibtex8,bibencoding=ascii,%
  language=auto,%
  style=numeric-comp,%
  %style=authoryear-comp, % Author 1999, 2010
  %bibstyle=authoryear,dashed=false, % dashed: substitute rep. author with ---
  sorting=nyt, % name, year, title
  maxbibnames=10, % default: 3, et al.
  %backref=true,%
  natbib=true % natbib compatibility mode (\citep and \citet still work)
}{biblatex}
    \usepackage{biblatex}

\PassOptionsToPackage{fleqn}{amsmath}       % math environments and more by the AMS
  \usepackage{amsmath}

% ********************************************************************
% General useful packages

\usepackage{graphicx} %
\usepackage{scrhack} % fix warnings when using KOMA with listings package
\usepackage{xspace} % to get the spacing after macros right
\PassOptionsToPackage{printonlyused,smaller}{acronym}
  \usepackage{acronym} % nice macros for handling all acronyms in the thesis
  %\renewcommand{\bflabel}[1]{{#1}\hfill} % fix the list of acronyms --> no longer working
  %\renewcommand*{\acsfont}[1]{\textsc{#1}}
  %\renewcommand*{\aclabelfont}[1]{\acsfont{#1}}
  %\def\bflabel#1{{#1\hfill}}
  \def\bflabel#1{{\acsfont{#1}\hfill}}
  \def\aclabelfont#1{\acsfont{#1}}
  
% ****************************************************************************************************
% 4. Setup floats: tables, (sub)figures, and captions

\usepackage{tabularx} % better tables
  \setlength{\extrarowheight}{3pt} % increase table row height
\newcommand{\tableheadline}[1]{\multicolumn{1}{l}{\spacedlowsmallcaps{#1}}}
\newcommand{\myfloatalign}{\centering} % to be used with each float for alignment
\usepackage{subfig}

% ****************************************************************************************************
% 5. Setup code listings

\usepackage{listings}
%\lstset{emph={trueIndex,root},emphstyle=\color{BlueViolet}}%\underbar} % for special keywords
\lstset{language=[LaTeX]Tex,%C++,
  morekeywords={PassOptionsToPackage,selectlanguage},
  keywordstyle=\color{RoyalBlue},%\bfseries,
  basicstyle=\small\ttfamily,
  %identifierstyle=\color{NavyBlue},
  commentstyle=\color{Green}\ttfamily,
  stringstyle=\rmfamily,
  numbers=none,%left,%
  numberstyle=\scriptsize,%\tiny
  stepnumber=5,
  numbersep=8pt,
  showstringspaces=false,
  breaklines=true,
  %frameround=ftff,
  %frame=single,
  belowcaptionskip=.75\baselineskip
  %frame=L
}



% ********************************************************************
% Fine-tune hyperreferences (hyperref should be called last)

\usepackage{hyperref}
\hypersetup{%
  %draft, % hyperref's draft mode, for printing see below
  colorlinks=true, linktocpage=true, pdfstartpage=3, pdfstartview=FitV,%
  % uncomment the following line if you want to have black links (e.g., for printing)
  %colorlinks=false, linktocpage=false, pdfstartpage=3, pdfstartview=FitV, pdfborder={0 0 0},%
  breaklinks=true, pageanchor=true,%
  pdfpagemode=UseNone, %
  % pdfpagemode=UseOutlines,%
  plainpages=false, bookmarksnumbered, bookmarksopen=true, bookmarksopenlevel=1,%
  hypertexnames=true, pdfhighlight=/O,%nesting=true,%frenchlinks,%
  urlcolor=CTurl, linkcolor=CTlink, citecolor=CTcitation, %pagecolor=RoyalBlue,%
  %urlcolor=Black, linkcolor=Black, citecolor=Black, %pagecolor=Black,%
  pdftitle={\myTitle},%
  pdfauthor={\textcopyright\ \myName, \myUni, \myFaculty},%
  pdfsubject={},%
  pdfkeywords={},%
  pdfcreator={pdfLaTeX},%
  pdfproducer={LaTeX with hyperref and classicthesis}%
}


% ********************************************************************
% Setup autoreferences (hyperref and babel)

% There are some issues regarding autorefnames
% http://www.tex.ac.uk/cgi-bin/texfaq2html?label=latexwords
% you have to redefine the macros for the
% language you use, e.g., american, ngerman
% (as chosen when loading babel/AtBeginDocument)
\makeatletter
\@ifpackageloaded{babel}%
  {%
    \addto\extrasamerican{%
      \renewcommand*{\figureautorefname}{Figure}%
      \renewcommand*{\tableautorefname}{Table}%
      \renewcommand*{\partautorefname}{Part}%
      \renewcommand*{\chapterautorefname}{Chapter}%
      \renewcommand*{\sectionautorefname}{Section}%
      \renewcommand*{\subsectionautorefname}{Section}%
      \renewcommand*{\subsubsectionautorefname}{Section}%
    }%
    \addto\extrasngerman{%
      \renewcommand*{\paragraphautorefname}{Absatz}%
      \renewcommand*{\subparagraphautorefname}{Unterabsatz}%
      \renewcommand*{\footnoteautorefname}{Fu\"snote}%
      \renewcommand*{\FancyVerbLineautorefname}{Zeile}%
      \renewcommand*{\theoremautorefname}{Theorem}%
      \renewcommand*{\appendixautorefname}{Anhang}%
      \renewcommand*{\equationautorefname}{Gleichung}%
      \renewcommand*{\itemautorefname}{Punkt}%
    }%
      % Fix to getting autorefs for subfigures right (thanks to Belinda Vogt for changing the definition)
      \providecommand{\subfigureautorefname}{\figureautorefname}%
    }{\relax}
\makeatother


% If you like the classicthesis, then I would appreciate a postcard.
% My address can be found in the file ClassicThesis.pdf. A collection
% of the postcards I received so far is available online at
% http://postcards.miede.de


% ****************************************************************************************************
% 0. Set the encoding of your files. UTF-8 is the only sensible encoding nowadays. If you can't read
% äöüßáéçèê∂åëæƒÏ€ then change the encoding setting in your editor, not the line below. If your editor
% does not support utf8 use another editor!
\PassOptionsToPackage{utf8}{inputenc}
  \usepackage{inputenc}

\PassOptionsToPackage{T1}{fontenc} % T2A for cyrillics
  \usepackage{fontenc}


% ****************************************************************************************************
% 1. Configure classicthesis for your needs here, e.g., remove "drafting" below
% in order to deactivate the time-stamp on the pages
% (see ClassicThesis.pdf for more information):

\PassOptionsToPackage{
  drafting=true,    % print version information on the bottom of the pages
  tocaligned=false, % the left column of the toc will be aligned (no indentation)
  dottedtoc=true,  % page numbers in ToC flushed right
  eulerchapternumbers=true, % use AMS Euler for chapter font (otherwise Palatino)
  linedheaders=false,       % chaper headers will have line above and beneath
  floatperchapter=true,     % numbering per chapter for all floats (i.e., Figure 1.1)
  eulermath=false,  % use awesome Euler fonts for mathematical formulae (only with pdfLaTeX)
  beramono=true,    % toggle a nice monospaced font (w/ bold)
  palatino=true,    % deactivate standard font for loading another one, see the last section at the end of this file for suggestions
  style=classicthesis % classicthesis, arsclassica
}{classicthesis}


% ****************************************************************************************************
% 2. Personal data and user ad-hoc commands (insert your own data here)

\newcommand{\myTitle}{Fading Illustration with Software Defined Radio\xspace}
\newcommand{\mySubtitle}{\xspace}
\newcommand{\myDegree}{\xspace}
\newcommand{\myName}{Halter Sara Cinzia \& Pross Naoki Sean\xspace}
\newcommand{\myProf}{Prof. Dr. Heinz Mathis\xspace}
\newcommand{\myOtherProf}{\xspace}
\newcommand{\mySupervisor}{Michel Nyffenegger\xspace}
\newcommand{\myFaculty}{Faculty of Electrical Engineeering\xspace}
\newcommand{\myDepartment}{Institute for Communication Systems ICOM\xspace}
\newcommand{\myUni}{University of Applied Sciences of East Switzerland OST\xspace}
\newcommand{\myLocation}{Rapperswil\xspace}
\newcommand{\myTime}{December 2021\xspace}
\newcommand{\myVersion}{Draft 1}

% Setup, finetuning, and useful commands
\newcommand{\ie}{i.\,e.}
\newcommand{\Ie}{I.\,e.}
\newcommand{\eg}{e.\,g.}
\newcommand{\Eg}{E.\,g.}

% ****************************************************************************************************
% 3. Loading some handy packages

% ********************************************************************
% Packages with options that might require adjustments

\PassOptionsToPackage{american}{babel} % change this to your language(s), main language last
    \usepackage{babel}

\usepackage{csquotes}
\PassOptionsToPackage{%
  %backend=biber,bibencoding=utf8, %instead of bibtex
  backend=bibtex8,bibencoding=ascii,%
  language=auto,%
  style=numeric-comp,%
  %style=authoryear-comp, % Author 1999, 2010
  %bibstyle=authoryear,dashed=false, % dashed: substitute rep. author with ---
  sorting=nyt, % name, year, title
  maxbibnames=10, % default: 3, et al.
  %backref=true,%
  natbib=true % natbib compatibility mode (\citep and \citet still work)
}{biblatex}
    \usepackage{biblatex}

\PassOptionsToPackage{fleqn}{amsmath}       % math environments and more by the AMS
  \usepackage{amsmath}

% ********************************************************************
% General useful packages

\usepackage{graphicx} %
\usepackage{scrhack} % fix warnings when using KOMA with listings package
\usepackage{xspace} % to get the spacing after macros right
\PassOptionsToPackage{printonlyused,smaller}{acronym}
  \usepackage{acronym} % nice macros for handling all acronyms in the thesis
  %\renewcommand{\bflabel}[1]{{#1}\hfill} % fix the list of acronyms --> no longer working
  %\renewcommand*{\acsfont}[1]{\textsc{#1}}
  %\renewcommand*{\aclabelfont}[1]{\acsfont{#1}}
  %\def\bflabel#1{{#1\hfill}}
  \def\bflabel#1{{\acsfont{#1}\hfill}}
  \def\aclabelfont#1{\acsfont{#1}}
  
% ****************************************************************************************************
% 4. Setup floats: tables, (sub)figures, and captions

\usepackage{tabularx} % better tables
  \setlength{\extrarowheight}{3pt} % increase table row height
\newcommand{\tableheadline}[1]{\multicolumn{1}{l}{\spacedlowsmallcaps{#1}}}
\newcommand{\myfloatalign}{\centering} % to be used with each float for alignment
\usepackage{subfig}

% ****************************************************************************************************
% 5. Setup code listings

\usepackage{listings}
%\lstset{emph={trueIndex,root},emphstyle=\color{BlueViolet}}%\underbar} % for special keywords
\lstset{language=[LaTeX]Tex,%C++,
  morekeywords={PassOptionsToPackage,selectlanguage},
  keywordstyle=\color{RoyalBlue},%\bfseries,
  basicstyle=\small\ttfamily,
  %identifierstyle=\color{NavyBlue},
  commentstyle=\color{Green}\ttfamily,
  stringstyle=\rmfamily,
  numbers=none,%left,%
  numberstyle=\scriptsize,%\tiny
  stepnumber=5,
  numbersep=8pt,
  showstringspaces=false,
  breaklines=true,
  %frameround=ftff,
  %frame=single,
  belowcaptionskip=.75\baselineskip
  %frame=L
}



% ********************************************************************
% Fine-tune hyperreferences (hyperref should be called last)

\usepackage{hyperref}
\hypersetup{%
  %draft, % hyperref's draft mode, for printing see below
  colorlinks=true, linktocpage=true, pdfstartpage=3, pdfstartview=FitV,%
  % uncomment the following line if you want to have black links (e.g., for printing)
  %colorlinks=false, linktocpage=false, pdfstartpage=3, pdfstartview=FitV, pdfborder={0 0 0},%
  breaklinks=true, pageanchor=true,%
  pdfpagemode=UseNone, %
  % pdfpagemode=UseOutlines,%
  plainpages=false, bookmarksnumbered, bookmarksopen=true, bookmarksopenlevel=1,%
  hypertexnames=true, pdfhighlight=/O,%nesting=true,%frenchlinks,%
  urlcolor=CTurl, linkcolor=CTlink, citecolor=CTcitation, %pagecolor=RoyalBlue,%
  %urlcolor=Black, linkcolor=Black, citecolor=Black, %pagecolor=Black,%
  pdftitle={\myTitle},%
  pdfauthor={\textcopyright\ \myName, \myUni, \myFaculty},%
  pdfsubject={},%
  pdfkeywords={},%
  pdfcreator={pdfLaTeX},%
  pdfproducer={LaTeX with hyperref and classicthesis}%
}


% ********************************************************************
% Setup autoreferences (hyperref and babel)

% There are some issues regarding autorefnames
% http://www.tex.ac.uk/cgi-bin/texfaq2html?label=latexwords
% you have to redefine the macros for the
% language you use, e.g., american, ngerman
% (as chosen when loading babel/AtBeginDocument)
\makeatletter
\@ifpackageloaded{babel}%
  {%
    \addto\extrasamerican{%
      \renewcommand*{\figureautorefname}{Figure}%
      \renewcommand*{\tableautorefname}{Table}%
      \renewcommand*{\partautorefname}{Part}%
      \renewcommand*{\chapterautorefname}{Chapter}%
      \renewcommand*{\sectionautorefname}{Section}%
      \renewcommand*{\subsectionautorefname}{Section}%
      \renewcommand*{\subsubsectionautorefname}{Section}%
    }%
    \addto\extrasngerman{%
      \renewcommand*{\paragraphautorefname}{Absatz}%
      \renewcommand*{\subparagraphautorefname}{Unterabsatz}%
      \renewcommand*{\footnoteautorefname}{Fu\"snote}%
      \renewcommand*{\FancyVerbLineautorefname}{Zeile}%
      \renewcommand*{\theoremautorefname}{Theorem}%
      \renewcommand*{\appendixautorefname}{Anhang}%
      \renewcommand*{\equationautorefname}{Gleichung}%
      \renewcommand*{\itemautorefname}{Punkt}%
    }%
      % Fix to getting autorefs for subfigures right (thanks to Belinda Vogt for changing the definition)
      \providecommand{\subfigureautorefname}{\figureautorefname}%
    }{\relax}
\makeatother
